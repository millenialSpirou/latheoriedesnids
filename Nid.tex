Pour Jolie c’est la dernière et sixième année à Montréal elle se
trouve au même nid depuis deux étés. Trois colloques, toutes
gentilles, le grille pain est efficace, il y a une petite galerie en
avant avec un set de patio éclectique, des tas de coussins et des
chaises adirondaques. \\[1ex]

C’est le début de l’été elle s’assoit sur l’un
des fauteuils, fait ses lectures en après-midi. Elle a apporté avec
elle dehors quelques volumes de poésie et des revues type
national-geographic avec des grandes photos de mammifères marins
immenses et paisibles et des chutes d’eau tropicales comme si c’était
le monde dans lequel on vivait. \\[1ex] 

La rue Casgrain lui fait face elle
prend une pause pour s’étirer une heure ou deux après s’être
réveillée, boit un café et fait du people watching en mangeant une
courge spaghetti. Elle range un peu les coussins, taponne le tout, un
bol de salade au couscous traîne quelque part, une dernière bouchée,
le soleil ne devrait pas tarder à s’éteindre. Depuis quatre ou cinq
mois c’est Cédric qui visite, plus jeune de quelques années, il est
mignon et gentil quelque peu naïf et anxieux, mais il séduit avec ses
yeux nuageux d’ailleurs, d’un peu plus loin.\\[1ex]

Il débarque de son vélo lui glisse un sourire s’assied a terre lui
demande de raconter sa journée. Il reste de la lumière ils en
profitent pour en faire de l’ellipse le temps ça se caresse ça se
domestique, on lui donne des commandes avec des biscuits et du
chocolat les minutes grésillent comme un bruit blanc le ciel délavé
vieux jeans. La chambre est à repeindre juste les bobettes à remettre
il en met partout il se tache et elle se fout de sa gueule il n’est
pas doué. La pizza est à terre Jolie aussi, assise en lotus la bière
aux lèvres.  Ça finit dans le lit, même si l’odeur de peinture c’est
pas génial c’est l’été faut bien se gâter se faire du bien. Ils se
promènent et mordillent les draps les draps volent Jolie chante. C’est
simple et collant, ils s’endorment, couchés en croix une tête sur le
ventre de l’autre, des oreillers qui traînent. Un peu de musique, ça
se mélange au vent et au ronronnement du fridge.\\

Elle a un soupir, le chien aussi. Les deux rient, ils s’endorment.
\clearpage
\begin{comment}
Cédric est un peu pathétique lui laisse des poèmes écrits en coin de
tables à côté du matelas au sol. Il continuera à en écrireElle dort un peu encore, c’est la
sieste, ce soir elle chante dans un bar. Ça la touche malgré tout ;
elle en garde quelques un par la suite, ils la suivent dans une petite
boite en carton, par exemple :\\

\setstretch{1.25}
\renewcommand{\baselinestretch}{1.25}
Avec tes taches de rousseur, poussières de feu\\
ça éclate tu es mon camion d’aube tu\\
verse dans le large une greffe de rayons\\
jette les murs pour des clairières\\
l’herbe haute l’air sec m’exfolie\\
le creux du sourire\\
‘ s’ouvre et on se berce hier s’arrête\\
demain commence après on verra\\
peut être\\
à petits pas\\
dors sans moi t’es bien\\
tu t-loves un peu dans les draps\\
d’une journée sans fin, ça s’étire \\
d’être de même, comme avars de paix \\
j’hallucine l’écrin je le sais\\
le vrai se condense pas\\
sur des brillants de douceur\\
Il faut que les vents fauchent de la scrape\\
l’amène dans les airs il faut\\
des noyaux pour que ça condense,\\
un grain de sel\\
une tache de poussière\\
tes taches de rousseur\\
\end{comment}
