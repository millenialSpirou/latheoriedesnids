\clearpage
\section*{marbre}
Les couleurs se versent dans leur tiédeurs ternes et
l'âme de Cédric se complait en épithètes chialeux. Le café est trop lent, il se
déploie dans la tasse, comme une routine de yogi au sourire imbécile, mielleux
et perdu mais avec quelque chose qui cloche derrière, une paix intérieure lactée
et donc trouble. La méditation n'est pas pour celui-ci, il manque de flexibilité
et ne peut dont pas s'assoir convenablement les jambes pliées.  Et méditer sur
une chaise, c'est con tout de même, on dirait qu'un principe essentiel est ainsi
transgressé. Et des principes ancestraux, il en a déjà transgressés assez ces
derniers temps.  Dans ce genre de mood il faut pas rester sur place, on
s'active, on va faire du sport, une bonne course dynamique pour se brasser les
os et ensuite hop la douche chaude et puis les étirements et un bon petit
poisson grillé, légumes vapeur le tout couronné d'un bon film, quelque chose de
réconfortant.  \\--- --- ou l'on fume. --- L'on fume si la morosité cynique est
cause révolutionnaire; la fuite du cliché aboutissant toujours et
inévitablement en cliché, en  clope et autres symboles phalliques.  Mais
tout de même, après tout, il faut bien meubler sa jeunesse. \\

Et d'ailleurs là où Cédric se trouvait, les meubles ne sont pas ce qui manque.
Ça alterne entre le contemporain lisse, le canapé ancien-régime, la bay window
entre deux vases chinois, on a droit à du granit, beaucoup de granit, et un bois
que l'on pourrait qualifier de japonais ; le rouge à lèvre recouvre
approximativement 30\% des lèvres avec goût ce qui est un ratio qui fonctionne
bien et ça indique à qui sont les drinks selon la teinte; ce qui permet de
remarquer le verre orphelin de Gallifée et de lui porter alors qu'elle contemple
paisiblement la rue McGill deux étages plus bas une cigarette à la main la
fenêtre légèrement ouverte, la fumée qui s'égare vers les bassins au bout du Vieux-Port. 
\\Le granit les talons les grands verres, très grands verres à vin, tout est
brillant et cristallin, avec de légères notes complémentaires de soyeux et de
velour, la pluie est légère et sophistiquée en glissant sur les grandes
fenêtres:

\columnratio{0.65} \setlength{\columnsep}{4em}
\begin{paracol}{2}
Cédric essaie
de s'extirper de sa bulle de poête cynique par le geste; il s'empare du
verre de Gallifée et essaie de se faufiler au travers de la piste de danse
improvisée, où les gens tournent et tournent et les grands talons font
tac-tac-tac et les grands verres cling cling, il bredouille un peu, aimerait
être plus souple dans le mouvement du corps, regrette de ne pas avoir appris
une danse sociale, la salsa problement, lorsqu'il était en Amérique Latine
avant d'entamer les études supérieures, il aurait peut-être eu le sang un
peu plus convivial. Il aurait dû être comme David et accepter la vie telle
qu'elle lui a été présentée au lieu de se morfondre en aphorismes à deux
piasses. 

\emph{Un cynisme comme une peau de lion pour cacher un amour fragile.}  \switchcolumn
\setstretch{1.2} \phantom{}
\small
\textit{"\textelp{}donc voilà ça a été un hiver un peu difficile pour moi au
    plan personnel, après l'histoire avec ma mère et j'avais besoin d'un peu de
    nouveau, ma job au début cà passait mais après \textelp{}"\\[3em]
    "\textelp{} C'est bon comme toune ça, tu aimes tu le hip hop progressiste,
    personellement je comprends mal l'anglais mais j'aime quand c'est engagé"}
\end{paracol}
Profitons des quelques
instants où Cédric s'avance le verre de Gallifée à la main vers
la fenêtre où cette dernière se berce au gré du vent d'automne pour faire un topo
rapide. 
\columnratio{0.35} \setlength{\columnsep}{4em}
\begin{paracol}{2}
\begin{rightcolumn} David est en train d'emménager avec Gallifée qui est
            toujours aussi empathique et chaleureuse dans un condo à Villeray
	    grâce à son salaire de consultant en \textit{art-investment}, 
    effleurer suptilement la hanche de Gallifée, amicalement bien sur, (pendant
    que son copain Dave raconte une vieille histoire d'universitaire à Joe
    histoire qui comprends une auberge de jeunesse, un bateau, et une
    omellette, 3 batons de dynamites, quelques cigares et un tigre asiatique et
    drogue, à risque de paraître vulgaire, \emph{évidemment} : drogue)
    et tirer un sourire peut-être un peu trop gras, mais il
    n'y a pas réflexion, il s'agit de réactions rapides. Tout ceci est
    confus et ça ne se choisit pas les sentiments, ni ceux bien tendres envers
    Gallifée ou ceux d'envie face à la situation de David. Ce genre de
    comportements ou de sentiments n'ont pas leur place au sein d'amitiés
    profondes qui ont
    l'âge d'un très vieux chien, quoique disons le,
    soyons \emph{honnêtes}, Gallifée est très, très  \\
\end{rightcolumn}
\begin{leftcolumn} \phantom\\\vspace{5em}
\setstretch{1.2} \phantom{}
\small
            \textit{"Oui je comprends comment tu te sens pour moi aussi ça a été
                difficile l'important c'est d'être ben relax, ensuite on s'en
                rend plus trop compte et c'est d'ailleurs très plaisant une fois
                qu'on se laisse allé, bon c'est sûr que c'est intimidant mais
        moi après en avoir parlé avec ma conjointe on s'est entendu qu'au final
c'est vraiment une question de confiance et d'honnêteté \textelp{}"}\\[10em]
\textit{"Écoute depuis que j'ai passé du Bikram ou Yin, je me sens telllement
	mieux, c'est comme plus passif, ça détend tout, jusqu'aux orteils, et
	maintenant eille je suis tellement plus productif, j'ai même reçu un
	bonus\ldots\\grâce au yoga, weird non""Ahhh ouin, effectivement, c'est
spécial"}
\end{leftcolumn}
\end{paracol}
\begin{center}\noindent\rule{0.5\textwidth}{0.4pt}\end{center}
Le café finit par couler, une fois la toast beurrée le matin peut
tranquillement  se résorber. On échange quelques bières dans un bar quelconque
car on est samedi après tout et on se ramasse par quelque mécanisme obscur dans
un grand immeuble vitré au vieux-port de Montréal, entre deux galleries trop
chères qui vendent plus du design graphique commercial léché que de l'art, que
l'on se retrouve à rigoler avec des petits regards admiratifs en coin ce qui
est quelque peu étrange d'ailleurs parce que David et Gallifée sont habitués à
l'endroit, pas précisément celui-ci mais son essence, son zeitgeist. Mais on ne
sort pas en ménage à trois, cela ne se fait pas, il faut comparses, bonhommie,
du léger, des personnages secondaires
à notre vie qui ont des catch phrase et ajoutent la bonne teneure de 
rocambolesque, il faut \emph{symétrie} donc il y a aussi Jean qui est ingénieur et
fait le tour du monde, il sort d'où on sait pu trop, la Zambie, toujours la
Zambie et la Malaysie surtout d'où il revient avec ses histoires
abracadabrantes, une légère barbe hirsute, de nouvelles normes culturelles et
une nouvelle personnalité qui vient se graffer sur ce qu'était Jean pré-nouveau
voyage qui change toujours mais toujours grand et blond et blanc, en fait tant
qu'à y être n'oublions pas d'appeler Joe pour qu'il se joigne à l'excursion vers
le party d'amis d'amis d'amis recursifs, Joe et  ses lunettes rondes et son humour
décapant, son charisme de dents tachées démontré lors de la marche du métro
vers l'édifice; il prend la peine de s'arrêter à chaque sortie de bar
pour s'introduire dans chaque discussion avec quelque présence féminine pour
en échapper un sobriquet un sourire lorsqu'il raconte une anecdote rapide
ou pousse un compliment, dents qui n'affectent pas son charisme
car il peut se le permettre avec ses cheveux gras et lisse, ses yeux sombres et
son teint olive, ses larges poignets ses yeux olive et son regard ombrageux,
son je-men-foutisme maintenant garni d'un concluant salaire à la radio de
Radio-Canada, d'ailleurs il ne se dirige pas vers les groupes de fumeurs
que pour cruiser pendant que ses amis l'attended en sirotant une bière à
la bouteille, il en profite aussi pour discuter de sujets épars, il en maîtrise
beaucoup grâce à son boulot, toujours en train de commenter tout.\\

Donc on monte un ascenseur au vieux-port un ascenseur qui fait zouuu tout en
douceur avec un cockpit comme si l'on voyageait dans un tube pneumatique et
on se taquine un peu, l'atmosphère est bien détendue, on est \emph{ben cocktail}.
Ça se remarque, on se dit quand même; entre deux feintes de boxes avec
Cédric Joe craque le mirroir qui lui fait dos sur quoi la joie et la
désapprobation sont totales (car le masculin, totaux, si laid) : "Eille Joe à soir casse pas toute caliss" --- "M'en
criss on Turnn Up\footnote{Vire fous, on fait le gros party, la teuf quoi} a
soir less go" "Joe\ldots J-J, tout-doux" --- "ouais d'accord Quoii
D'AUtres".  Donc on monte dans ce tube et ça fait zouuu et on giggle entre
quelques gorgées partagées de vin blanc à la bouteille. Et l'on cogne entre
deux simagrées à cette grande porte lisse et pleine. On entre dans ce loft
mezzanine dont les deux étages donnent sur une immense fenêtre qui elle
donne sur le centre-ville illuminé et le fleuve qui s'allonge.  Bien évidemment
il y a du trap, un mobilier de jeunesse flétrie--disons fin vingtaine à fin
trentaine--riche, bon rien de dynastique mais tout de même, en 2018, le
mobilier d'une telle cohorte \emph{nécessite} le trap.  \footnote{Le trap est un style musical qui a ses origines dans le
        hip-hop du sud des états-unis. Il est marqué par de très rapides coups
        de snare en triplettes sur de larges basses lines qui ondulent sous le
        rythme de gros gras kick-drum.  Le tout est garnit alors de
        \textit{mumble rap}, un style de rap où l'artiste déploie paresseusement
        ses rhymes, lorsqu'il y en a, avec l'accent d'un ivrogne sur la codéine,
        le rythme encore en triplettes: tatata-tatata-tatata-TA.  Nous pourrions
        qualifier ce dernier style d'une série de dactyles punchés à la fin par
un anapeste moderne} 

Le loft est situé au dernier étage d'un nouvel immeuble, les planchers de granit peut-être, on admire
le tout en se délaissant de son imperméable et en enlevant ses botillons
mais quelqu'un nous enfarge: Jean est ben trop high pour délacer ses souliers
polis ou pour avoir une quelconque appréciation esthétique soutenue 
qu'il se trémousse déjà en se faisant aller les bras vers la partie
plus sombre de l'endroit où le dance floor a été méticuleusement déposé, et
Joe, Joe cherche déjà les verres et n'en a rien à foutre vraiment des bâtisses,
il cherche des
verres surtout pour se chercher un verre parce que la bière ça fait pas la job
et il a judicieusement ammené un fiable 26oz de Jim Bean 
\begin{comment}Le Jim
    Bean est un whiskey, un bourbon pour être plus précis, connu comme étant
    typiffiant de l'amérique avec un gros r sale, d'une toxicité masculine, avec
    sa bouteille nettement carrée et son petit coup de coude en fin de gorgée,
il est pas mal quand même.  Et pour le prix, pour le prix\ldots 
\end{comment}

On est dans la cuisine, on prend place, se cherche un verre, se présente
aux divers convives qui étaient déjà présents, certains pour un verre d'eau
d'autres pour fumer sous la hoote, ou encore, comme c'est le cas de Salomé
simplement pour s'éloigner de la fête parce que déjà à cette heure pas si
tardive  ça se tortille, ça fait de la grosse poudre, ça s'ostine sur la
prochaine toune, il y a à ce que l'on peut comprendre déjà eu tout 
un combat de masculinité toxique, pas aux poings mais un est parti en claquant
la porte, une histoire de poker ou d'ex on ne sait plus. 

Alors Cédric décide d'arpenter les lieux et se déplace vers les escaliers 
en évitant des conversations sur la vie, l'amour et la crise financière,
les danseurs un peu trop enjoués et finalement il peut faire l'ascension du 
colimaçon en bois, celui-ci nettement québécois, du frêne recyclé on dirait,
et il arrive à un cercle de petites conversations sur les fauteuils rouges
amples mais angulaires joliment installés en ménage à trois sur le
bord de la rampe. Il faut socialiser au final,
on ne reste pas entre petites cliques comme de gros quebz salles à un party, on
mingle, \emph{caliss}. On fait des
rencontres inopinées
\\

[Note de l'auteur : dialogue émotif à ajouter]\\[1em]

\begin{paracol}{2}
	\setstretch{1.2} 

\phantom
\small
\textit{"\textelp{} Faut vraiment qu'on aille au Charlevoix cet hiver
il y a un rave avec un line up de DJ de fou mon gars. Un truc de malade. 
Et ensuite BIM, on s'enfile des tartiflettes, le ricard, un bon flanc, 
et on se la met bien rigo, on revient de la teuf en chien de trainaux
et tout ça va être décalquant"}
\switchcolumn
\setstretch{1.8}
avec, évidemment, la vue majestueuse sur la deuxième moitié en hauteur de la bay
window, cette lumière colorée à travers les échancrures des grands luminaires
abstrait de glissants d'étincelles.  
A sa gauche il y a une \emph{salle à poud}, la chambre en temps normal 
destinée aux vacanciers
américains ou français qui déboursent quelques centaines de dollars par nuit
pour l'escapade et on rentre dans cette pièce
et en fait il y a un miroir bien positionné,\end{paracol}
la vitre vers le haut, un miroir
sans cadre, pour gratouiller tout ce qui reste sans que ça coince dans les
craques, scratch scratch l'âme de rasoir et évidemment, lorsqu'on s'en fait
proposer une tite ligne, et qu'on est là pour relaxer, et que c'est un nom de
la politique bien connu maintenant, connu pour ses opinions plutôt radicales
gauchistes, qui vous proposent la dite tite ligne, alors on dit mais oui en fait
allons-y. Alors Cédric prend place dans le cercle ou plutôt rectangle courbé
de chaises en aluminium et fait un signe de tête et un gentil "Salut". 
D'ailleur juste à côté on retrouve Joe qui roucoule comme un
perroquet et fait des becs dans le coup à une animatrice
de variété autrefois connue qui a d'ailleur disparu plutôt brusquement de la
sphère médiatique Québécoise, petit fait divers intéressant bien vite résolu par
l'animatrice entre deux sniffées, elle est \emph{en thèse} , elle en avait marre
des médias et de la superficialité; elle est retournée aux études comme elle
l'explique en ce moment, en \emph{thèse} sur le poète Brézilien Carlos Drummond
Andrade et sa démarche formelle face à la langue populaire, \\ on a plus les
animatrices de variété qu'on avait\ldots

\begin{center}\noindent\rule{0.5\textwidth}{0.4pt}\end{center} Les petites
heures approchent et il se retourne à contempler la vie et Salomé, la jeune
femme avocate sincère et spirituelle qui lui fait face dans la cuisine entre le
fridge et le comptoir auquel elle est indolemment accotée. Il voudrait lui
contempler les bas-fonds de l'âme et s'y plonger, mais les heures sont petites,
ses yeux sont vitreux, la musique se fait longue et plate.  Il fixe un
ustensile, n'écoute rien, ni ce qu'elle dit ni le bruit
de fond constant ni les paroles du rapper \textit{Lil-Mickey-Royce}. Il lance
quelques regards autour de lui pour constater une étrange apathie, et il
faudrait percer l'air et rejoindre Salomé ou quelqu'un quelque chose.
Regards croisés, une discussion authentique? On se voit s'ouvrir à cette
belle étrangère qui nous expose un intéressant dilemme éthique dans le droit
international. Faire une vraie rencontre et prendre rendez-vous, pour une
marche sur le Mont-Royal, avec un chien, c'est l'automne, c'est coloré. Mais
elle parle dans le néant, il se retourne,
plonge sa main gauche dans un gros bol de
cheetos et pendant qu'elle élabore sur la constitutionnalité post-moderne;
il se liche un à un, lentement, chaque doigt de la main gauche. \\


Joe est probablement déjà rentré avec quelqu'un(e) il ne pourra donc pas
remonter le moral à Cédric avec quelques jokes de mononc bien tournées et des
gesticulations (c'est sa seule utilité)

Cédric s'avance le verre de vin à la main, verre toujours
taché du rouge à lèvres sobres de Gallifée, en boit un grand trait et le dépose sur
une corniche car la fenêtre est ouverte et donne sur un faux balcon. Jean et Joe
cassent quelque chose de vitré en dansant, si on peut appeler cela de
la danse à cette heure-ci, c'est plutôt un rassemblement amateur de danseurs
du ventre. David vient rejoindre Cédric à la fenêtre, lui tend une bière.
Les deux prennent une gorgé, haussent les épaules. Le premier fait à l'autre un
signe de tête. Ils sortent et descendent les escaliers.\\

Une fois
sur le trottoir de la grande
rue McGill avec ses nouveaux lampadaires chics et sa belle asphalte large et
ondulée et les commerces de luxe ils se dirigent lentement vers le port en
allumant un joint. 

Arrivé à la promenade derrière à la piste cyclable ils s'avancent vers la
fin d'un pier, comme une presqu'île pittoresque. \\ Ils prennent place à un
banc, râlent contre les conneries de la vie, quelques vicissitudes partagées
malgré leurs parcours divergents. Ouvrent chacun une cannette de Old Milwaukee,
par nostalgie de l'adolescence, David humecte la colle d'un autre joint alors que
son ami s'essoufle d'un soupir mélancolique mais paisible.
\\ -- Pis Dave, tu
penses tu que ça va ressembler à ça votre loft une fois retaper pis toute\\ 
-- Non dude, voyons, j'ai tu vraiment l'air d'un gars qui plaque des
reproductions de
Jackson 
Pollock partout \\
-- Ben non Comon jte niaise\\
-- Je sais mais ça hit fort quand même de voir du monde de même avec qui
t'as jamais eu tant que ça en commun et te dire, ben oui ce serait logique,
ce serait moi dans pas long tout ça \textelp{} Pis toi, t'a fini ta maîtrise tu
vas tu au Doc?\\
-- Je sais pas trop encore, ça pu l'air trop pertinent, j'ai l'impression
de juste ingérer des bits d'informations, style oie à fois gras\\
-- Je t'entends, même vibe pour moi quand j'ai fini par finir l'école\\

\textelp{} et au fait, maintenant que j'y pense, pour votre appart là, vous
avez pas aussi commandé le même genre de comptoir contemporain en granit
messemble\\
-- C'est pas du granit, \emph{criss}, c'est du \emph{marbre}\\

Cédric humecte maintenant le joint qui lui est repassé en le tournant entre son
pouce et son index, déposant la salive avec son auriculaire à l'extrémité du
cherry, il s'émouvoit encore un peu du paysage, urbain mais intime quand
même\ldots quelques rares passants, la lumière du port, une eau trouble et
miroitante.\\[1ex]
Il décide qu'il est maintenant impératif de séduire Gallifée; préférablement sur un comptoir.
\clearpage

