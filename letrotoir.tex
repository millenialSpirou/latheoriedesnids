Jade et Cédric marchent sur la rue St-hubert. Arrivés au coin Beaubien, par là
que les commerces commencent à border la rue, ils croisent Fernando et Carlos.
Le premier de Guinée, le deuxième du Brésil, ils demandent quelque information
de touriste; comme où aller prendre un verre. Cédric voit la une occasion de
pratiquer son portugais; langue qu'il avait entâmé d'apprendre après son
décrochage d'études polytechniciennes d'ingénieur, il invitent donc ces deux
nouveaux acolytes à les suivres au Notre dame des quilles, établissement réputé
pour son ouverture d'esprits et ses tendances alternatives.

Ils marchent 2 a deux, largeur du trottoir obligeant. Cedric et Fernando
discutent litterature, ce sont deux programmeurs d'ordinateurs dont la veritable
passion est la litterature, c'est une revelation pour eux deux de se retrouve si
proche mentalement ainsi que geographiquement, malgre les continents qui les
separent.

Lorsqu'ils arrivent au bar ils prennent place au comptoir en L. Fernando et
Cedric continuent leur discussion littéraire alors que Carlos et Jade s'effacent
sur le trait inférieur du L. Ces deux premiers partagent la même idée de la vie,
écrire du code informatique parce que ça se vend, alors que la poésie, pff,
personne ne paye pour celà.

Cédric surveille du coin de l'oeil Jade et Carlos, il a l'air, sinon de la
dérenger d'être au moins, irritant. Elle se lève après quelque dizaine de minute
pour venir jouer avec Cédric, comment le fait-elle? Et bien elle a l'air d'aimer
lui mettre les mains dans la figure, se retourne danse dos à ventre sur lui,
bref, des simagrées. Cédric continue tant bien que mal sa conversation avec
Fernando

En sortant Jade précise à son ami platonique qu'elle n'aime pas le
compère Brésilien, il y a une note dans sa voix qui trahi comme une
espèce de connaissance de l'individu que l'on aurait pas prédit.

Le groupe se resepare deux a deux, on se promet de se revoir;
pour ce faire cedric a ajoute fernando comme ami sur facebook.
On voit ainsi qu'il travaille pour la fondation tomas sankara et
pour linux international, un peu de googlage serverait bien; mais
a premiere vue il s'agit la d'un organisme visant a favoriser l'education
sur l'informatique en afrique de l'ouest.
